 %!TeX root = Project 1 Math 5.tex
\documentclass{article}
\usepackage[dvipsnames, svgnames, x11names]{xcolor}
\usepackage{tikz}
\usepackage{pgfplots}
\usepackage{setspace}
\usepackage{units}
\usepackage{graphicx}
\usepackage{amsopn}
\usepackage{bbding}
\usepackage{amsmath}
\usepackage{hyperref}
\usepackage{cancel}
\usepackage{gensymb}
\usepackage[margin = 1in]{geometry}
\title{Project 1, Transformations}
\date{25 March, 2025}
\author{Tejas Patel}
\begin{document}
\maketitle
\section{}
\textbf{Matrices defined}: $A=\begin{bmatrix}1&2\\3&4\end{bmatrix},\;\;B=\begin{bmatrix}0&2&-1\\1&1&10\end{bmatrix},\;\; V=\begin{bmatrix}1&4&0\end{bmatrix}$
\\[0.1in]When \verb+show(A), show(B), show(V)+ was input into the python terminal, the output from SageMath was $\begin{bmatrix}1&2\\3&4\end{bmatrix},\;\;\begin{bmatrix}0&2&-1\\1&1&10\end{bmatrix},\;\; \begin{bmatrix}1&4&0\end{bmatrix}$ respectively. The matrices were repeated back to me as I originially entered them
\\[0.01in]$$\text{For }QQ \rightarrow RR \Rightarrow \begin{bmatrix}1.00000000000000&3.00000000000000\\2.00000000000000&4.00000000000000\end{bmatrix}$$ 
$$\text{For }QQ \rightarrow RDF  \Rightarrow \begin{bmatrix}1.0 & 2.0 \\3.0 & 4.0\end{bmatrix}$$ 
\\[0.1in] $RR$ and $RDF$ seem to be data types for the matrices. $RR$ appears to be an aribtrary precision floating point, outputting the maximum number of decimal places allowed. It won't be as precise as $QQ$, which is symbolic math, nor will it be as fast as $RDF$, which seems to be reduced decimal form, but it will get you a precise decimal answer when youre looking for one.
\\[0.1in]$$\text{RREF}(B)=\begin{bmatrix}
    1 & 0 & \frac{21}{2} \\
0 & 1 & -\frac{1}{2}
\end{bmatrix}$$
\\[0.1in] The command \verb+I4 = identity matrix(n)+ creates an identity matrix of the degree of the input \verb+n+. 
\\[0.1in]For the original command: $\begin{bmatrix}1 & 0 & 0 & 0 \\0 & 1 & 0 & 0 \\0 & 0 & 1 & 0 \\0 & 0 & 0 & 1\end{bmatrix}$ 
For input = 2: $\begin{bmatrix}1 & 0 \\0 & 1\end{bmatrix}$
For input = 5: $\begin{bmatrix}1 & 0 & 0 & 0 & 0 \\0 & 1 & 0 & 0 & 0 \\0 & 0 & 1 & 0 & 0 \\0 & 0 & 0 & 1 & 0 \\0 & 0 & 0 & 0 & 1\end{bmatrix}$
\pagebreak
\section{}
\textbf{New matrices defined}: $C=\begin{bmatrix}3 & 7 \\1 & 6\end{bmatrix}$
\\[0.05in]\textbf{a}: In SageMath, to add: \verb|A+C| results in $\begin{bmatrix}4 & 9 \\4 & 10\end{bmatrix}$, to subtract \verb|A-C| results in $\begin{bmatrix}-2 & -5 \\2 & -2\end{bmatrix}$
\\[0.05in]Lastly, to multiply: \verb|A*C| results in $\begin{bmatrix}5 & 19 \\13 & 45\end{bmatrix}$, bonus: \verb+det(A*C)+ results in -22
\\\textbf{b:} $$B^T=\begin{bmatrix}0 & 1 \\2 & 1 \\-1 & 10\end{bmatrix}\text{ shows that } \forall x \in B \left\{x_{ij} \rightarrow x_{ji} \right\} $$
\\[0.1in] \verb+A*C==C*A+ returned \verb+false+, meaning $AC\neq CA$
\\[0.1in] \verb+(A*C).transpose()==A.transpose()*C.transpose()+ returned \verb+false+, meaning $(AC)^T\neq A^TC^T$
\\[0.1in] \verb|(A+C).transpose()==A.transpose()+C.transpose()| returned \verb+true+, meaning $(A+C)^T=A^T+C^T$
\\[0.1in] \verb|(c*A).transpose()==c*A.transpose()| returned \verb|true| for $c=6$ and $c=10$, meaning $(cA)^T=c*A^T$
\\[0.1in] The left number of columns in the left matrix must equal the number of rows in the right matrix. If $m$ is the number of columns in A and $p$ is the number of rows in B, then $m=p$ must be true. The output of matrix multiplication is going to be of dimension $n \times q$
\section{}
\textbf{All tikzpictures are transformations of vector V=[1,4]}
\\a: The output from the code is 4 graphs, each one containing 2 vectors. The blue arrows represent $P1, P2, P3, P4$ and the purple arrows represent $Q1,Q2,Q3,Q4$. See the output tikzpictures \color{blue}\hyperref[sec:intro]{here}\color{black}
\\[0.05in]b: Reflects across the $y$ axis (or $x_2$ aixs). Vector $(2,-4)$ becomes $(-2,-4)$ \color{blue}\hyperref[sec:b]{tikzpicture}\color{black}
\\[0.05in]c: Reflects across $y=x$, $(3,7)$ becomes $(7,3)$ and $(1,0)$ becomes $(0,1)$ \color{blue}\hyperref[sec:c]{tikzpicture}\color{black}
\\[0.05in]d: Reflects across $y=-   x$, $(3,7)$ becomes $(-7,-3)$ and $(1,0)$ becomes $(0,-1)$ \color{blue} \hyperref[sec:d]{tikzpicture}\color{black}
\\[0.05in]e: Maintains $x$ value, sets $y$ value to $0$. $(3,7)$ becomes $(3,0)$ and $(0,1)$ becomes $(0,0)$ \color{blue}{\hyperref[sec:e]{tikzpicture}}\color{black}
\\[0.05in]f: Pretty much using the reciprocal of the columns of M, the matrix $\begin{bmatrix}1&0\\0&-1\end{bmatrix}$ resolves the reflection across the $x_1$ axis, where $(3,7)$ becomes $(-3,7)$
\\[0.05in]g: The transformation $\begin{bmatrix}\cos\theta&-sin\theta\\\sin\theta&\cos\theta\end{bmatrix}$ rotates the vector it is multiplied by by angle $\theta$ counterclockwise. $(3,7)$ becomes approx. $(-2.828, 7.071)$, creating a $45\degree$ dihedral angle
\pagebreak
\section{}
\subsection{Derivation}
If $\begin{bmatrix}x_1&x_2\end{bmatrix}\begin{bmatrix}a\\b\end{bmatrix} = a+b$, then $x_1a+x_2b=a+b$, meaning $x_1=1,~x_2=1$
\\[0.05in]If $\begin{bmatrix}x_3&x_4\end{bmatrix}\begin{bmatrix}a\\b\end{bmatrix} = a-b$, then $x_3a+x_4b=a-b$, meaning $x_3=1,~x_4=-1$
\\Combining these together, we get transformation matrix \boxed{A=T=\begin{bmatrix}1&1\\1&-1\end{bmatrix}}
\subsection{Check/Proof}
Checking $\begin{bmatrix}1&1\\1&-1\end{bmatrix}$ with the Identity Matrix $\begin{bmatrix}1&0\\0&1\end{bmatrix}$. 
The expected answer is $\begin{bmatrix}1+0\\0-1\end{bmatrix} = \begin{bmatrix}1\\-1\end{bmatrix}$
\\[0.2in]$$\begin{bmatrix}1&1\\1&-1\end{bmatrix}\begin{bmatrix}1\\0\end{bmatrix}= \begin{bmatrix}
1\cdot 1 + 1 \cdot 0 \\ 1\cdot 0+-1\cdot 1\end{bmatrix}=\begin{bmatrix}1+0\\0-1\end{bmatrix}=\begin{bmatrix}1\\-1\end{bmatrix}$$
We can observe the answer matches the expected, showing the transformation matrix $A$ is the correct transformation. The answer was checked with SageMath
\section{Determining the Transformation Matrix J}
$\begin{bmatrix}
x_{11}&x_{12}\\x_{21}&x_{22}\\x_{31}&x_{32}\\x_{41}&x_{42}
\end{bmatrix}\begin{bmatrix}x_1\\x_2\end{bmatrix}=\begin{bmatrix}
    2x_1 - 5x_2\\0\\x_1+4x_2\\x_2
\end{bmatrix}$
\subsection{Determining individual rows}
Row 1: $x_1x_{11}+x_2x_{12}=2x_1-5x_2 \Rightarrow x_{11}=2\text{ and }x_{12}=-5$
\\[0.05in]Row 2: $x_1x_{21}+x_2x_{22}=0 \Rightarrow x_{21}=0\text{ and }x_{22}=-0$
\\[0.05in]Row 3: $x_1x_{31}+x_2x_{32}=x_1+4x_2 \Rightarrow x_{31}=1\text{ and }x_{32}=4$
\\[0.05in]Row 4: $x_1x_{41}+x_2x_{42}=x_2 \Rightarrow x_{41}=0\text{ and }x_{42}=1$
\subsection{Combining the results}
Combining the results back into the original matrix form:
$J=\begin{bmatrix}2&-5\\0&0\\1&4\\0&1\end{bmatrix}$
\pagebreak
\section{Section 3 tikzpictures}
\label{sec:intro}
\scalebox{0.5}{\input{a.pgf} \input{a2.pgf}}
\\\scalebox{0.5}{\input{a3.pgf} \input{a4.pgf}}
\pagebreak
\subsection{b}
\label{sec:b} 
{\input{b.pgf}} 
\pagebreak
\subsection{c}
\label{sec:c}
{\input{c.pgf}}
\pagebreak
\subsection{d}
\label{sec:d}
{\input{d.pgf}}
\pagebreak
\subsection{e}
\label{sec:e}
{\input{e.pgf}}


\end{document}